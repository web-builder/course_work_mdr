\newpage
\chapter*{СПИСОК ОБОЗНАЧЕНИЙ}
\addcontentsline{toc}{chapter}{СПИСОК ОБОЗНАЧЕНИЙ}
WWW (World Wide Web) — Всемирная паутина

W3C (World Wide Web Consortium) — Консорциум Всемирной паутины

DOM (Document Object Model) — объектная модель документа

XML (eXtensible Markup Language) — расширяемый язык разметки

MathML (Mathematical Markup Language) — язык математической разметки

\TeX{} (TEX) — система компьютерной вёрстки

ePub (Electronic Publication) — открытый формат электронных версий книг

Java — объектно-ориентированный, структурный, императивный, кроссплатформенный язык программирования

КПК — Карманный персональный компьютер 

\newpage
\chapter*{ВВЕДЕНИЕ}
\addcontentsline{toc}{chapter}{ВВЕДЕНИЕ}
С появлением технологии отображения информации имитирующей обычную печать на бумаге, неуклонно растет популярность узкоспециализированных компактных планшетных компьютерных устройств (электронных книг).

Несмотря на высокие темпы роста, в настоящее время, все еще сохраняются проблемы с поддержкой технической и научной литературы современными электронными устройствами для чтения. Преобладающим методом распространения научной информации является графическое представление, обладающее рядом негативных факторов: отсутствием возможности для поиска и индексации, сложным созданием документов, зачастую плохим качеством изображений.

Одной из причин слабой поддержки технической и научной литературы является отсутствие универсального механизма, удобного для создания, редактирования и отображения информации такого рода.

Целями данной работы являются:
\begin{enumerate}
 \item Разработка формата компьютерных данных, способного хранить любые математические выражения и простую текстовую информацию.
 \item Проектирование наиболее важных и объемных компонентов приложения для визуализации содержимого нового формата на экранах узкоспециализированных компактных планшетных компьютерных устройств.
 \item Создание прототипа части компонентов приложения.
\end{enumerate}

Для достижения поставленной цели необходимо решить следующие задачи:
\begin{enumerate}
 \item[1)] Проанализировать существующие наработки по данной проблеме.
 \item[2)] Спроектировать приложение для электронных устройств поддерживающее новый формат данных.
 \item[3)] Исследовать необходимые меры для успешного внедрения и популяризации нового формата данных и приложения.
\end{enumerate}


\newpage
\chapter{ФОРМАТ ДАННЫХ}

\section{Обзор существующих технологий}
\subsection{Система \TeX{}}
\TeX{} --– система компьютерной вёрстки, разработанная американским профессором информатики Дональдом Кнутом в целях создания компьютерной типографии. В неё входят средства для секционирования документов, для работы с перекрёстными ссылками. Многие считают \TeX{} лучшим способом для набора сложных математических формул\cite{Kotelnikov}.

Стиль работы при подготовке текста в системе \TeX{} отличается от стиля работы при работе с редактором Microsoft Word и ближе к программированию, чем к редактированию текста в обычном смысле. Система TEX разделяет более абстрактное представление текста в исходном файле и его типографское расположение на странице, так что в принципе вся работа может совершаться с абстрактным представлением текста и завершаться компиляцией лишь в самом конце. Тем не менее, если мощность компьютера позволяет, рекомендуется часто компилировать текст, чтобы вовремя замечать и устранять неизбежные при его наборе погрешности и ошибки.

Разделение абстрактного представления текста в исходном файле и его типографского исполнения имеет как достоинства, так и недостатки. Несомненным достоинством является то, что основное внимание автора сосредотачивается на содержании текста, а аспекты его форматирования (выбор шрифта, детали расположения текста и набора формул и т.~п.) частично передоверяются тщательно разработанным алгоритмам самой системы, а частично могут быть выбраны автором уже после того, как содержание текста создано. (Подобный подход применяется и в редакторе Microsoft Word при последовательном использовании стилей форматирования документа.) Недостатком же системы \TeX{} является необходимость работать со значительно более абстрактным, лишенным наглядности представлением текстом, что осложняет первоначальное освоение системы и на этапе подготовки текста нередко приводит к ошибкам, выявляющимся лишь при компиляции исходного файла.

\TeX{} можно использовать для всех видов текста любой сложности. Многие издательства используют его для книгопечатания или книжного набора..

Пример записи формулы  на языке \TeX{} для нахождения действительных корней квадратного уравнения:

\begin{equation}
 \verb!x = \frac{-b \pm \sqrt{b^2 - 4ac}}{2a}!,  
\end{equation} 
эта формула отобразится в виде:
\begin{equation}
 x = \frac{-b \pm \sqrt{b^2 - 4ac}}{2a}.\label{eq:2}
\end{equation}

\subsection{Язык  MathML}

MathML – язык математической разметки, основанный на технологии XML.

Текущая спецификация MathML 3.0, утвержденная в октябре 2010 года,  отвечает современным идеям математической системы представления, предоставляя возможность использовать высокоразвитую систему математической нотации в различных документах.

MathML рассматривает не только представление, но и смысл элементов формулы. Разрабатывается система разметки математической семантики OpenMath, призванная дополнить MathML.  

XML-структура MathML обеспечивает широкую область использования и позволяет быстро отображать формулы в WWW, а также легко интерпретировать их значения в математических программных продуктах.

Для соответствия различным требованиям научного сообщества MathML разрабатывается с учетом следующих критериев [http://www.w3.org/Math/]:
\begin{enumerate}
 \item Предоставление информации, подходящей как для обучения, так и для научной коммуникации любого типа.
 \item Возможность преобразования данных с другими математическими форматами, как презентационными, так и семантическими.
 \item Поддержка корректного просмотра длинных выражений.
 \item Обеспечение расширяемости.
 \item Поддержка шаблонов и других средств редактирования математической информации.
\end{enumerate}


Пример записи формулы  на языке разметки MathML для нахождения действительных корней квадратного уравнения (в коротком виде):

\begin{equation}
 \verb|x= {-b± sqrt {{b} ^ {2} -4ac}} over {2a}|,
\end{equation}
и в полной форме (????????? это две формы масМЛ или что это?):
\begin{verbatim}
<?xml version="1.0" encoding="UTF-8"?>
<math xmlns="http://www.w3.org/1998/Math/MathML">
<mrow>
  <mi>x</mi>
  <mo>=</mo>
  <mfrac>
    <mrow>
      <mrow>
        <mo>-</mo>
        <mi>b</mi>
        <mo>±</mo>
      </mrow>
      <msqrt>
        <mrow>
          <msup>
            <mi>b</mi>
            <mn>2</mn>
          </msup>
          <mo>-</mo>
          <mrow>
            <mn>4</mn>
            <mi>a</mi>
            <mi>c</mi>
          </mrow>
        </mrow>
      </msqrt>
    </mrow>
    <mrow>
      <mn>2</mn>
      <mi>a</mi>
    </mrow>
  </mfrac>
 </mrow>
</math>
\end{verbatim}
которая будет иметь тот же вид, что~\eqref{eq:2}.

\section{Определение нового формата данных}
Решением поставленной задачи по разработке необходимого формата данных может быть слияние формата ePub с языком математической разметки MathML.

ePub — открытый формат электронных версий книг, разработанный Международным форумом по цифровым публикациям.

Формат данных ePub позволяет производить и распространять цифровую публикацию в одном файле, обеспечивая совместимость между программным и аппаратным обеспечением, необходимым для воспроизведения цифровых книг и других публикаций с плавающей вёрсткой.

Новый формат данных будет представлять собой контейнер включающий в себя:
\begin{enumerate}
 \item XML-файл, содержащий математические выражения описанные на языке MathML.
 \item XHTML-файл с текстовой информацией.
 \item XML-файл с описанием данных о документе, автора и условий распространения документа.
 \item Папка с векторной и растровой графикой.
 \item Папка со шрифтами, что позволит дополнять документ необходимыми дополнительными шрифтами.
 \item Файл таблицы стилей.
\end{enumerate}
%% ВАЖНО
%% http://www.reeed.ru/info_epub.php
%% принцип отображения формул на картинке в статье?!
%% 

\newpage
\chapter{ПРОЕКТИРОВАНИЕ ПРИЛОЖЕНИЯ}
\section{Архитектура}
\subsection{Состояния}

\subsection{Классы}

\subsection{Последовательность}

\subsection{Кооперация}

\subsection{Компоненты}

\subsection{Развертывание}

\section{?}

\newpage
\chapter{ВНЕДРЕНИЕ РЕШЕНИЯ}
\section{Конвертер}

\section{Версия поддерживающая шифрование}

\newpage
\chapter*{ЗАКЛЮЧЕНИЕ}
\addcontentsline{toc}{chapter}{ЗАКЛЮЧЕНИЕ}

%%// NO FOLLOW
В ходе исследования текущей темы проанализированы основы технологий предназначенных для описания математических выражений, некоторые из популярных форматов электронных версий книг.
Качественная визуализация математических выражений сложна по причине их сложной и высокоразвитой двумерной символьной системы обозначений.
Были проанализированы существующие подходы, методики и наработки по данной проблеме.

\newpage
\chapter*{Список использованных источников}
\addcontentsline{toc}{chapter}{Список использованных источников}

%%но тут будем делать по другому я чуть позже сделаю

Список использованных источников
1. Википедия – свободная энциклопедия [Электронный ресурс]. -  http://ru.wikipedia.org/wiki/MathML . - (дата обращения: 13.05.2013).

2. Википедия – свободная энциклопедия [Электронный ресурс]. -  http://ru.wikipedia.org/wiki/TeX . - (дата обращения: 14.05.2013).

%3. Википедия – свободная энциклопедия [Электронный ресурс]. -  http://ru.wikipedia.org/wiki/Document_Object_Model . - (дата обращения: 10.05.2013). 

4. Википедия – свободная энциклопедия [Электронный ресурс]. -  http://ru.wikipedia.org/wiki/XML . - (дата обращения: 15.05.2013).

5. W3C – Консорциум Всемирной паутины [Электронный ресурс]. -  http://www.w3.org/Math/. - (дата обращения: 15.05.2013).